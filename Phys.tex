\section{Физика}
\subsection{Механика}
\subsubsection{Уравнение Эйлера-Лагранжа}
Лагранжиан:
$$L(q_{i},\dot q_{i},t)=T-V$$
$T$ - кинетическая энергия системы\\
$V$ - потенциальная энергия системы\\
n - число степеней свободы\\\\
Уравнение Эйлера-Лагранжа для $n$ степеней совободы:
\begin{equation*}
 \begin{cases}
   $$\frac{d}{dt}\frac{\partial L}{\partial \dot q_{1}}-\frac{\partial L}{\partial q_{1}} = 0$$\\
   $$\frac{d}{dt}\frac{\partial L}{\partial \dot q_{2}}-\frac{\partial L}{\partial q_{2}} = 0$$\\
   ...\\
   $$\frac{d}{dt}\frac{\partial L}{\partial \dot q_{n}}-\frac{\partial L}{\partial q_{n}} = 0$$
 \end{cases}
\end{equation*}

\subsubsection{Интегралы движения}
Уравнение движения имеет вид:
$$\ddot q = F(q,\dot q,t)$$
Для $n$ степеней свободы:
$$q = q(q_{1},...,q_{n})$$
$q$ - обобщенный вектор координат
\begin{equation*}
 \begin{cases}
   $$\ddot q_{1} = F(q_{1},...,q_{n},\dot q_{1},...,\dot q_{n},t)$$\\
   $$\ddot q_{2} = F(q_{1},...,q_{n},\dot q_{1},...,\dot q_{n},t)$$\\
   ...\\
   $$\ddot q_{n} = F(q_{1},...,q_{n},\dot q_{1},...,\dot q_{n},t)$$
 \end{cases}
\end{equation*}
При решении данной системы дифференциальных уравнений второго порядка получим решение:
$$q=q(t,C_{1},C_{2},...,C_{2n})$$
$C_{1},C_{2},...,C_{2n}$ - константы, называемые интегралами движения.\\\\
Три "главных" интеграла движения:
\begin{enumerate}
    \item Время однородно - уравнение движения системы не зависит от начального момента времени
    $$L=L(q,\dot q)\longrightarrow \frac{dL}{dt}=0$$
    $$\sum_{i}\frac{\partial L}{\partial \dot q_{i}}\dot q-L=const=E$$
    Полная энергия замкнутой системы не меняется;
    \item Пространство однородно - уравнение движения системы сохраняется при смещении ее в пространстве (При отсутствии влияния внешних полей и объектов)
    $$L=L(t,q)$$
    $$\sum_{i}\frac{\partial L}{\partial \dot q_{i}}=const=\textbf{P}$$
    Закон сохранения импульса;
    \item Пространство изотропно - при повороте системы вокруг некой оси уравнение движения не меняется.\\
    $\overline{\varphi}$ - вектор угла поворота системы (перпендикулярен плоскости вращения, по модулю равен углу поворота)\\
    Изменения обобщенных координат и скорости выражаются через векторные произведения:
    \begin{equation*}
     \begin{cases}
       $$\delta q=[\delta\overline{\varphi},q]$$\\
       $$\delta\dot q=[\delta\overline{\varphi},\dot q]$$
     \end{cases}
    \end{equation*}
    $$\delta L = \sum_{i} \frac{\partial L}{\partial q_{i}}\delta q_{i} + \frac{\partial L}{\partial \dot q_{i}}\delta \dot q_{i} =\left\{\frac{\partial L}{\partial q_{i}} = \dot p_{i}, \frac{\partial L}{\partial \dot q_{i}} = p_{i} \right\}= \delta \overline{\varphi}\frac{d}{dt}\sum_{i}\textbf{q}_{i}\times \textbf{p}_{i}=0$$
    $$\sum_{i}\textbf{q}_{i}\times \textbf{p}_{i} = const = \textbf{L}$$
    Закон сохранения момента импульса
\end{enumerate}


\subsection{Общая теория колебаний}